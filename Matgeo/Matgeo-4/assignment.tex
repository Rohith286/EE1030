\let\negmedspace\undefined
\let\negthickspace\undefined
\documentclass[journal]{IEEEtran}
\usepackage[a5paper, margin=10mm, onecolumn]{geometry}
%\usepackage{lmodern} % Ensure lmodern is loaded for pdflatex
\usepackage{tfrupee} % Include tfrupee package

\setlength{\headheight}{1cm} % Set the height of the header box
\setlength{\headsep}{0mm}     % Set the distance between the header box and the top of the text

\usepackage{gvv-book}
\usepackage{gvv}
\usepackage{cite}
\usepackage{amsmath,amssymb,amsfonts,amsthm}
\usepackage{algorithmic}
\usepackage{graphicx}
\usepackage{textcomp}
\usepackage{xcolor}
\usepackage{txfonts}
\usepackage{listings}
\usepackage{enumitem}
\usepackage{mathtools}
\usepackage{gensymb}
\usepackage{comment}
\usepackage[breaklinks=true]{hyperref}
\usepackage{tkz-euclide} 
\usepackage{listings}
% \usepackage{gvv}                                        
\def\inputGnumericTable{}                                 
\usepackage[latin1]{inputenc}                                
\usepackage{color}                                            
\usepackage{array}                                            
\usepackage{longtable}                                       
\usepackage{calc}                                             
\usepackage{multirow}                                         
\usepackage{hhline}                                           
\usepackage{ifthen}                                           
\usepackage{lscape}
\begin{document}

\bibliographystyle{IEEEtran}
\vspace{3cm}

\title{CHAPTER - 1\\Vector Arithmetic}
\author{EE24BTECH11061 - Rohith Sai}
% \maketitle
% \newpage
% \bigskip
{\let\newpage\relax\maketitle}

\renewcommand{\thefigure}{\theenumi}
\renewcommand{\thetable}{\theenumi}
\setlength{\intextsep}{10pt} % Space between text and floats

\numberwithin{figure}{enumi}
\renewcommand{\thetable}{\theenumi}


\section{1.10 Unit Vector}
\begin{enumerate}
\item [1.10.5] If $\vec{a} =\vec{i+j+2k}$ and $\vec{b} = \vec{2i+j-2k}$, find the unit vector in the direction of
\begin{enumerate}
    \item $6\vec{a}$
    \item $2\vec{a}-\vec{b}$
\end{enumerate}
\textbf{Solution:}
\begin{table}[h!]    
      \centering
      \begin{center}
    \begin{tabular}{|c|c|c|} 
        \hline
            \textbf{Variable} & \textbf{Description} & \textbf{Value} \\ 
        \hline
            $\vec{a}$    & Length of $BC$ & $6cm$ \\ 
        \hline
            $\vec{b}$    & Length of $AC$ & $?$\\ 
        \hline
            $\vec{c}$    & Length of $AB$ & $5cm$\\
        \hline
            $\angle{ABC}$& Angle $B$      & $60\degree$\\
        \hline
    \end{tabular}
\end{center}
      \caption{}
    \end{table}\\
a) According to the question:
\begin{align}
    6\vec{a} = 6\myvec{1\\1\\2},\\
    \abs{\abs{6\vec{a}}} = 6\abs{\abs{\vec{a}}} = 6\sqrt{{\brak{\vec{a}}^\top\brak{\vec{a}}}},\\
    \implies \abs{\abs{6\vec{a}}} = 6\sqrt{6}
\end{align}
We know that, to find the unit vector:
\begin{align}
    \implies \frac{6\vec{a}}{\abs{\abs{6\vec{a}}}} = \frac{1}{6\sqrt{6}}\myvec{6\\6\\12}
\end{align}
Therefore, the required unit vector is:
\begin{align}
    \implies \frac{1}{\sqrt{6}}\myvec{1\\1\\2}
\end{align}

b) According to the question:
\begin{align}
    2\vec{a}-\vec{b} = 2\myvec{1\\1\\2}-\myvec{2\\1\\-2} = \myvec{0\\1\\6},\\
    \abs{\abs{2\vec{a}-\vec{b}}} = \sqrt{{\brak{2\vec{a}-\vec{b}}^\top\brak{2\vec{a}-\vec{b}}}},\\
    \implies \abs{\abs{2\vec{a}-\vec{b}}} = \sqrt{37}
\end{align}
We know that, to find the unit vector:
\begin{align}
    \implies \frac{2\vec{a}-\vec{b}}{\abs{\abs{2\vec{a}-\vec{b}}}} = \frac{1}{\sqrt{37}}\myvec{0\\1\\6}
\end{align}
Therefore, the required unit vector is:
\begin{align}
    \implies \frac{1}{\sqrt{37}}\myvec{0\\1\\6}
\end{align}
\begin{figure}[htp]
    \centering
    \includegraphics[width=10cm]{figs/figure.png}
    \label{fig:figure}
\end{figure}
\end{enumerate}
\end{document}