\let\negmedspace\undefined
\let\negthickspace\undefined
\documentclass[journal]{IEEEtran}
\usepackage[a5paper, margin=10mm, onecolumn]{geometry}
%\usepackage{lmodern} % Ensure lmodern is loaded for pdflatex
\usepackage{tfrupee} % Include tfrupee package

\setlength{\headheight}{1cm} % Set the height of the header box
\setlength{\headsep}{0mm}     % Set the distance between the header box and the top of the text

\usepackage{gvv-book}
\usepackage{gvv}
\usepackage{cite}
\usepackage{amsmath,amssymb,amsfonts,amsthm}
\usepackage{algorithmic}
\usepackage{graphicx}
\usepackage{textcomp}
\usepackage{xcolor}
\usepackage{txfonts}
\usepackage{listings}
\usepackage{enumitem}
\usepackage{mathtools}
\usepackage{gensymb}
\usepackage{comment}
\usepackage[breaklinks=true]{hyperref}
\usepackage{tkz-euclide} 
\usepackage{tikz}
\usepackage{listings}
\usetikzlibrary{patterns}
% \usepackage{gvv}                                        
\def\inputGnumericTable{}                                 
\usepackage[latin1]{inputenc}                                
\usepackage{color}                                            
\usepackage{array}                                            
\usepackage{longtable}                                       
\usepackage{calc}                                             
\usepackage{multirow}                                         
\usepackage{hhline}                                           
\usepackage{ifthen}                                           
\usepackage{lscape}
\begin{document}
\bibliographystyle{IEEEtran}
\vspace{3cm}

\title{GATE\\PH - 2012}
\author{EE24BTECH11061 - Rohith Sai}
\maketitle

\renewcommand{\thefigure}{\theenumi}
\renewcommand{\thetable}{\theenumi}

\section*{Single Correct 1 Mark each}
\begin{enumerate}
\item Identify the CORRECT statement for the following vectors $\vec{a = 3i+ 2j}$ and $\vec{b} = i + 2j$
\begin{multicols}{2}
    \begin{enumerate}
        \item The vectors $\vec{a}$ and $\vec{b}$ are linearly independent
        \item The vectors $\vec{a}$ and $\vec{b}$ are linearly dependent
        \item The vectors $\vec{a}$ and $\vec{b}$ are orthogonal
        \item The vectors $\vec{a}$ and $\vec{b}$ are normalized
    \end{enumerate}
\end{multicols}

\item Two uniform thin rods of equal length, $L$, and masses $M_1$ and $M_2$ are joined together along the
length. The moment of inertia of the combined rod of length $2L$ about an axis passing through the
mid-point and perpendicular to the length of the rod is,
\begin{multicols}{2}
    \begin{enumerate}
        \item $\brak{M_1 + M_2}\frac{L^2}{12}$
        \item $\brak{M_1 + M_2}\frac{L^2}{6}$
        \item $\brak{M_1 + M_2}\frac{L^2}{3}$
        \item $\brak{M_1 + M_2}\frac{L^2}{2}$
    \end{enumerate}
\end{multicols}
    
\item The space-time dependence of the electric field of a linearly polarized light in free space is given by $xE_0 \cos{\omega t - kz}$ where $E_0$, $\omega$ and $k$ are the amplitude, the angular frequency and the wavevector, respectively. The time averaged energy density associated with the electric field is
\begin{multicols}{2}
    \begin{enumerate}
        \item $\frac{1}{4} \epsilon_0 E_0^2$
        \item $\frac{1}{2} \epsilon_0 E_0^2$
        \item $\epsilon_0 E_0^2$
        \item $2 \epsilon_0 E_0^2$
    \end{enumerate}
\end{multicols}

\item If the peak output voltage of a full wave rectifier is 10 V, its d.c. voltage is
\begin{multicols}{2}
    \begin{enumerate}
        \item 10.0 V
        \item 7.07 V
        \item 6.36 V
        \item 3.18 V
    \end{enumerate}
\end{multicols}

\item A particle of mass $m$ is confined in a two dimensional square well potential of dimension $a$. This
potential $V\brak{x,y}$ is given by
\begin{align*}
    V\brak{x,y} = 0 & \text{ for } -a < x < a \text{ and } -a < y < a\\
    = \infty &  \text{ elsewhere}
\end{align*}
The energy of the first excited state for this particle is given by,
\begin{multicols}{2}
    \begin{enumerate}
        \item $\frac{\pi^2 \hbar^2}{m a^2}$
        \item $\frac{2 \pi^2 \hbar^2}{m a^2}$
        \item $\frac{5 \pi^2 \hbar^2}{2 m a^2}$
        \item $\frac{4 \pi^2 \hbar^2}{m a^2}$
    \end{enumerate}
\end{multicols}

\item The isothermal compressibility, $\kappa$ of an ideal gas at temperature $T_0$ and volume $V_0$, is given by
\begin{multicols}{2}
    \begin{enumerate}
        \item $-\frac{1}{V_0} \frac{\partial V}{\partial P} \Big|_{T_0}$
        \item $\frac{1}{V_0} \frac{\partial V}{\partial P} \Big|_{T_0}$
        \item $-V_0 \frac{\partial P}{\partial V} \Big|_{T_0}$
        \item $V_0 \frac{\partial P}{\partial V} \Big|_{T_0}$
    \end{enumerate}
\end{multicols}

\item The ground state of sodium atom $\brak{^{11}Na}$ is a $^{2} S_{1/2}$ state. The difference in energy levels arising in the presence of a weak external magnetic field $B$, given in terms of Bohr magneton, $\mu_B$, is
\begin{multicols}{2}
    \begin{enumerate}
        \item $\mu_B$ 
        \item $2 \mu_B$
        \item $4 \mu_B$
        \item $6 \mu_B$
    \end{enumerate}
\end{multicols}

\item For an ideal Fermi gas in three dimensions, the electron velocity $v_F$ at the Fermi surface is related
to electron concentration $n$ as, 
\begin{multicols}{2}
    \begin{enumerate}
        \item $v_F \propto n^{2/3}$
        \item $v_F \propto n$
        \item $v_F \propto n^{1/2}$
        \item $v_F \propto n^{1/3}$
    \end{enumerate}
\end{multicols}

\item Which one of the following sets corresponds to fundamental particles?
\begin{multicols}{2}
    \begin{enumerate}
        \item proton, electron and neutron
        \item proton, electron and photon
        \item electron, photon and neutrino
        \item quark, electron and meson
    \end{enumerate}
\end{multicols}

\item In case of a Geiger-Muller (GM) counter, which one of the following statements is CORRECT?
\begin{multicols}{2}
    \begin{enumerate}
        \item Multiplication factor of the detector is of the order of $10^{10}$
        \item Type of the particles detected can be identified
        \item Energy of the particles detected can be distinguished
        \item Operating voltage of the detector is few tens of Volts
    \end{enumerate}
\end{multicols}

\item A plane electromagnetic wave traveling in free space is incident normally on a glass plate of
refractive index $\frac{3}{2}$. If there is no absorption by the glass, its reflectivity is
\begin{multicols}{2}
    \begin{enumerate}
        \item $4 \%$
        \item $16 \%$
        \item $20 \%e$
        \item $50 \%$
    \end{enumerate}
\end{multicols}

\item A Ge semiconductor is doped with acceptor impurity concentration of $10^{15}$ atoms/cm$^3$. For the given hole mobility of 1800 cm$^{2}$/V-s, the resistivity of this material is
\begin{multicols}{2}
    \begin{enumerate}
        \item $0.288 \Omega$ cm
        \item $0.694 \Omega$ cm
        \item $3.472 \Omega$ cm
        \item $6.944 \Omega$ cm
    \end{enumerate}
\end{multicols}

\item A classical gas of molecules, each of mass $m$, is in thermal equilibrium at the absolute temperature, $T$. The velocity components of the molecules along the Cartesian axes are $v_x$, $v_y$ and $v_z$. The mean value of $\brak{v_x + v_y}^2$ is
\begin{multicols}{2}
    \begin{enumerate}
        \item $\frac{k_B T}{m}$
        \item $\frac{3}{2} \frac{k_B T}{m}$
        \item $\frac{1}{2} \frac{k_B T}{m}$
        \item $2\frac{k_B T}{m}$
    \end{enumerate}
\end{multicols}
\end{enumerate}
\end{document}