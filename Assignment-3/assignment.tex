%iffalse
\let\negmedspace\undefined
\let\negthickspace\undefined
\documentclass[journal,12pt,twocolumn]{IEEEtran}
\usepackage{cite}
\usepackage{amsmath,amssymb,amsfonts,amsthm}
\usepackage{algorithmic}
\usepackage{graphicx}
\usepackage{textcomp}
\usepackage{xcolor}
\usepackage{txfonts}
\usepackage{listings}
\usepackage{enumitem}
\usepackage{mathtools}
\usepackage{gensymb}
\usepackage{comment}
\usepackage[breaklinks=true]{hyperref}
\usepackage{tkz-euclide} 
\usepackage{listings}
\usepackage{gvv}                                        
%\def\inputGnumericTable{}                                 
\usepackage[latin1]{inputenc}                                
\usepackage{color}                                            
\usepackage{array}                                            
\usepackage{longtable}                                       
\usepackage{calc}                                             
\usepackage{multirow}                                         
\usepackage{hhline}                                           
\usepackage{ifthen}                                           
\usepackage{lscape}
\usepackage{tabularx}
\usepackage{array}
\usepackage{float}


\newtheorem{theorem}{Theorem}[section]
\newtheorem{problem}{Problem}
\newtheorem{proposition}{Proposition}[section]
\newtheorem{lemma}{Lemma}[section]
\newtheorem{corollary}[theorem]{Corollary}
\newtheorem{example}{Example}[section]
\newtheorem{definition}[problem]{Definition}
\newcommand{\BEQA}{\begin{eqnarray}}
\newcommand{\EEQA}{\end{eqnarray}}
\newcommand{\define}{\stackrel{\triangle}{=}}
\theoremstyle{remark}
\newtheorem{rem}{Remark}

% Marks the beginning of the document
\begin{document}
\bibliographystyle{IEEEtran}
\vspace{3cm}

\title{CHAPTER - 20\\Vector Algebra and\\Three Dimensional Geometry}
\author{EE24BTECH11061 - Rohith Sai}
\maketitle
\newpage
\bigskip

\renewcommand{\thefigure}{\theenumi}
\renewcommand{\thetable}{\theenumi}

\section{Fill in the Blanks}
\begin{enumerate}
\item Let $OA=a, OB = 10a + 2b$ and $OC =b$ where $O, A$ and $C$ are non-collinear points. Let $p$ denote the area of the quadrilateral $OABC$, and let $q$ denote the area of the parallelogram with $OA$ and $OC$ as adjacent sides. If $p=kq$, then $k$=.......
\hfill{\textbf{(1997 - 2 Marks)}}
\end{enumerate}

\section{True/False}
\begin{enumerate}
\item Let $\vec{A}, \vec{B}$ and $\vec{C}$ be unit vectors suppose that $\vec{A}.\vec{B} = \vec{A}.\vec{C}=0$, and that the angle between $\vec{B}$ and $\vec{C}$ is $\frac{\pi}{6}$. Then $\vec{A}=\pm2(\vec{B}\times\vec{C})$.\\
\hfill {\textbf{(1981 - 2 Marks)}}

\item If $\vec{X}.\vec{A}=0, \vec{X}.\vec{B}=0, \vec{X}\vec{C}=0$ for some non-zero vector $\vec{X}$, then $[A B C]=0$
\hfill \textbf{(1983 - 1 Mark)}

\item The points with position vectors $\vec{a+b}$, $\vec{a-b}$ and $\vec{a+kb}$ are collinear for all real values of $k$.
\hfill{\textbf{(1984 - 1 Mark)}}

\item For any three vectors $\vec{a}, \vec{b}$ and $\vec{c}$, $(\vec{a}-\vec{b}).{(\vec{b}-\vec{c)}\times(\vec{c}-\vec{a})} = 2\vec{a}.(\vec{b}\times\vec{c}).$\\
\hfill{\textbf{(1989 - 1 Mark)}}
\end{enumerate}

\section{MCQs with One Correct Answer}
\begin{enumerate}
\item The scalar $\vec{A}.(\vec{B}+\vec{C})\times(\vec{A}+\vec{B}+\vec{C})$ equals:\\
\begin{enumerate}[label=(\alph*)]
\item  $0$
\item $[\vec{A}\ \vec{B}\ \vec{C}] + [\vec{B}\ \vec{C}\ \vec{A}]$
\item $[\vec{A}\ \vec{B}\ \vec{C}]$
\item None of these
\end{enumerate}
\hfill{\textbf{(1981 - 2 Marks)}}

\item For non-zero vectors $\vec{a}, \vec{b}, \vec{c} |(\vec{a}\times\vec{b}).\vec{c}| = |\vec{a}||\vec{b}||\vec{c}|$ holds if and only if
\begin{enumerate}[label=(\alph*)]
\item $\vec{a}.\vec{b}=0 ,\ \vec{b}.\vec{c}=0$
\item $\vec{b}.\vec{c}=0 ,\ \vec{c}.\vec{a}=0$
\item $\vec{c}.\vec{a}=0 ,\ \vec{a}.\vec{b}=0$
\item $\vec{a}.\vec{b}=\ \vec{b}.\vec{c}=\ \vec{c}.\vec{a}=0$
\end{enumerate}
\hfill{\textbf{(1982 - 2 Marks)}}

\item The volume of the parallelopiped whose sides are given by $\vec{OA}=\vec{2i}-\vec{2j} ,\ \vec{OB} = \vec{i}+\vec{j}-\vec{k} ,\ \vec{OC}= \vec{3i}-\vec{k}$, is 
\begin{enumerate}[label = (\alph*)]
\item $\frac{4}{13}$
\item $4$
\item $\frac{2}{7}$
\item None of these
\end{enumerate}
\hfill{\textbf{(1983 - 1 Mark)}}

\item The points with position vectors $\vec{60i + 3j},\ \vec{40i-8j},\ \vec{ai-52j}$ are collinear if
\begin{enumerate}[label = (\alph*)]
\item $a=-40$
\item $a=40$
\item $a=20$
\item None of these
\end{enumerate}
\hfill{\textbf{(1983 - 1 Mark)}}

\item Let $\vec{a}, \vec{b}, \vec{c}$ be three non coplanar vectors and $\vec{p}, \vec{q},\vec{r}$ are vectors defined by the relations $\vec{p}=\frac{\vec{b}\times\vec{c}}{[\vec{a}\ \vec{b}\ \vec{c}]}, \vec{q}=\frac{\vec{c}\times\vec{a}}{[\vec{a}\ \vec{b}\ \vec{c}]},\vec{r}=\frac{\vec{a}\times\vec{b}}{[\vec{a}\ \vec{b}\ \vec{c}]}$ then the value of the expression $(\vec{a}+\vec{b}).\vec{p}+(\vec{b}+\vec{c}).\vec{q}+(\vec{c}+\vec{a}).\vec{r}$ is equal to
\begin{enumerate}[label = (\alph*)]
\item $0$
\item $1$
\item $2$
\item $3$
\end{enumerate}
\hfill{\textbf{(1988 - 2 Marks)}}

\item Let $a, b, c$ be distinct non-negative numbers. If the vectors $\vec{ai + aj + ck},\ \vec{i+k}$ and $\vec{ci+cj+bk}$ lie in a plane, then $c$ is
\begin{enumerate}[label=(\alph*)]
\item the Arithmetic Mean of $a$ and $b$
\item the Geometric Mean of $a$ and $b$
\item the Harmonic Mean of $a$ and $b$
\item equal to zero
\end{enumerate}
\hfill{\textbf{(1993 - 1 Mark)}}

\item Let $\vec{p}$ and $\vec{q}$ be the position vectors of $P$ and $Q$ respectively, with respect to $O$ and $|\vec{p}| = p$, $|\vec{q}| = q$. The points $R$ and $S$ divide $PQ$ internally and externally in the ratio $2:3$ respectively. If $OR$ and $OS$ are perpendicular then
\begin{enumerate}[label = (\alph*)]
\item $9p^2 =4q^2$
\item $4p^2 = 9q^2$
\item $9p = 4q$
\item $4p = 9q$
\end{enumerate}
\hfill{\textbf{(1994)}}

\item Let $\alpha,\ \beta,\ \gamma$ be distinct real numbers. The points with position vectors $\vec{\alpha i+ \beta j + \gamma k}$, $\vec{\beta i+ \gamma j+ \alpha k}$, $\vec{\gamma i + \alpha j + \beta k}$
\begin{enumerate}[label = (\alph*)]
\item are collinear
\item form an equilateral triangle
\item form a scalene triangle
\item form a right angles triangle
\end{enumerate}
\hfill{\textbf{(1994)}}

\item Let $\vec{a=i-j}$, $\vec{b=j-k}$, $\vec{c=k-i}$. If $\vec{d}$ is a unit vector such that $\vec{a}.\vec{d} = 0 = [\vec{b} \ \vec{c} \ \vec{d}]$, then $\vec{d}$ equals
\begin{enumerate}[label=(\alph*)]
\item $\pm \frac{\vec{i+j-2k}}{\sqrt{6}}$
\item $\pm \frac{\vec{i+k-k}}{\sqrt{3}}$
\item $\pm \frac{\vec{i+j+k}}{\sqrt{3}}$
\item $\pm \vec{k}$
\end{enumerate}
\hfill{\textbf{(1995S)}}

\item If $\vec{a},\vec{b},\vec{c}$ are non coplanar unit vectors such that $\vec{a}\times(\vec{b} \times \vec{c}) = \frac{(\vec{b}+\vec{c})}{\sqrt{2}}$, then the angle between $\vec{a}$ and $\vec{b}$ is
\begin{enumerate}[label = (\alph*)]
\item $\frac{3 \pi}{4}$
\item $\frac{\pi}{4}$
\item $\frac{\pi}{2}$
\item $\pi$
\end{enumerate}
\hfill{\textbf{(1995S)}}
\end{enumerate}

\end{document}