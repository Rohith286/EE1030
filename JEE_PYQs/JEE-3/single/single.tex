\let\negmedspace\undefined
\let\negthickspace\undefined
\documentclass[journal]{IEEEtran}
\usepackage[a5paper, margin=10mm, onecolumn]{geometry}
%\usepackage{lmodern} % Ensure lmodern is loaded for pdflatex
\usepackage{tfrupee} % Include tfrupee package

\setlength{\headheight}{1cm} % Set the height of the header box
\setlength{\headsep}{0mm}     % Set the distance between the header box and the top of the text

\usepackage{gvv-book}
\usepackage{gvv}
\usepackage{cite}
\usepackage{amsmath,amssymb,amsfonts,amsthm}
\usepackage{algorithmic}
\usepackage{graphicx}
\usepackage{textcomp}
\usepackage{xcolor}
\usepackage{txfonts}
\usepackage{listings}
\usepackage{enumitem}
\usepackage{mathtools}
\usepackage{gensymb}
\usepackage{comment}
\usepackage[breaklinks=true]{hyperref}
\usepackage{tkz-euclide} 
\usepackage{listings}
% \usepackage{gvv}                                        
\def\inputGnumericTable{}                                 
\usepackage[latin1]{inputenc}                                
\usepackage{color}                                            
\usepackage{array}                                            
\usepackage{longtable}                                       
\usepackage{calc}                                             
\usepackage{multirow}                                         
\usepackage{hhline}                                           
\usepackage{ifthen}                                           
\usepackage{lscape}
\begin{document}
\bibliographystyle{IEEEtran}
\vspace{3cm}

\title{JEE MAINS 2023\\ April 13 - Shift 1}
\author{EE24BTECH11061 - Rohith Sai}
\maketitle

\renewcommand{\thefigure}{\theenumi}
\renewcommand{\thetable}{\theenumi}

\section*{Single Correct }

\begin{enumerate}
\item For $x \in \mathbb{R}$, two real valued functions $f\brak{x}$ and $g\brak{x}$ are such that, $g\brak{x} = \sqrt{x}+1$  and $fog\brak{x} = x+3-\sqrt{x}$. Then $f\brak{0}$ is equal to
\begin{multicols}{2}
    \begin{enumerate}
        \item 5
        \item 0
        \item -3
        \item 1
    \end{enumerate}
\end{multicols}

\item Let the equation of plane passing through the line of intersection of the planes $x+2y+az=2$ and $x-y+z=3$ be $5x-11y+bz = 6a-1$. For $c \in \mathbb{Z}$, the distance of this plane from the point $\brak{a, -c, c}$ is $\frac{2}{\sqrt{a}}$, then $\frac{a+b}{c}$ is equal to
\begin{multicols}{2}
    \begin{enumerate}
        \item -4
        \item 2
        \item -2
        \item 4
    \end{enumerate}
\end{multicols}

\item Fractional part of the number $\frac{4^{2022}}{15}$ is  equal to
\begin{multicols}{2}
    \begin{enumerate}
        \item $\frac{4}{15}$
        \item $\frac{8}{15}$
        \item $\frac{1}{15}$
        \item $\frac{14}{15}$
    \end{enumerate}
\end{multicols}

\item Let $y=y_1\brak{x}$ and $y=y_2\brak{x}$ be the solution curves of the differential equation $\frac{dy}{dx} = y+7$ with initial conditions $y_1\brak{0} = 0$ and $y_2\brak{0} = 1$ respectively. Then the curves $y=y_1\brak{x}$ and $y=y_2\brak{x}$ intersect at
\begin{multicols}{2}
    \begin{enumerate}
        \item no point
        \item infinite number of points
        \item one point
        \item two points
    \end{enumerate}
\end{multicols}

\item The area of the region enclosed by the curve $f\brak{x} = max\{\sin{x}, \cos{x}\}$, $-\pi\leq x \leq\pi$ and the x-axis is
\begin{multicols}{2}
    \begin{enumerate}
        \item $2\sqrt{2}\brak{\sqrt{2}+1}$
        \item $4\sqrt{2}$
        \item $4$
        \item $2\brak{\sqrt{2}+1}$
    \end{enumerate}
\end{multicols}
\end{enumerate}
\end{document}