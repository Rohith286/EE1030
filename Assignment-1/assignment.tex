%iffalse
\let\negmedspace\undefined
\let\negthickspace\undefined
\documentclass[journal,12pt,twocolumn]{IEEEtran}
\usepackage{cite}
\usepackage{amsmath,amssymb,amsfonts,amsthm}
\usepackage{algorithmic}
\usepackage{graphicx}
\usepackage{textcomp}
\usepackage{xcolor}
\usepackage{txfonts}
\usepackage{listings}
\usepackage{enumitem}
\usepackage{mathtools}
\usepackage{gensymb}
\usepackage{comment}
\usepackage[breaklinks=true]{hyperref}
\usepackage{tkz-euclide} 
\usepackage{listings}
\usepackage{gvv}                                        
%\def\inputGnumericTable{}                                 
\usepackage[latin1]{inputenc}                                
\usepackage{color}                                            
\usepackage{array}                                            
\usepackage{longtable}                                       
\usepackage{calc}                                             
\usepackage{multirow}                                         
\usepackage{hhline}                                           
\usepackage{ifthen}                                           
\usepackage{lscape}
\usepackage{tabularx}
\usepackage{array}
\usepackage{float}


\newtheorem{theorem}{Theorem}[section]
\newtheorem{problem}{Problem}
\newtheorem{proposition}{Proposition}[section]
\newtheorem{lemma}{Lemma}[section]
\newtheorem{corollary}[theorem]{Corollary}
\newtheorem{example}{Example}[section]
\newtheorem{definition}[problem]{Definition}
\newcommand{\BEQA}{\begin{eqnarray}}
\newcommand{\EEQA}{\end{eqnarray}}
\newcommand{\define}{\stackrel{\triangle}{=}}
\theoremstyle{remark}
\newtheorem{rem}{Remark}

% Marks the beginning of the document
\begin{document}
\bibliographystyle{IEEEtran}
\vspace{3cm}

\title{CHAPTER-16\\Applications of Derivatives}
\author{EE24BTECH11061 - Rohith Sai}
\maketitle
\newpage
\bigskip

\renewcommand{\thefigure}{\theenumi}
\renewcommand{\thetable}{\theenumi}

\section{E. Subjective Problems}

\begin{enumerate}

\item A point $P$ is given on the circumference of a circle of radius $r$. Chord $QR$ is parallel to the tangent at P. Determine the maximum possible area of the triangle $PQR.$

\hfill (1990 - 4 Marks)


\item A window of perimeter $P$ (including the base of the arch) is in the form of a rectangle surrounded by a semi circle. The semi-circular portion is fitted with clear glass transmits three times as much light per square meter as the coloured glass does.\\What is the ratio for the sides of the rectangle so that the window transmits the maximum light?
\hfill (1991 - 4 Marks)

\item A cubic $f(x)$ vanishes at $x=2$ and has a relative minimum/maximum at $x=-1$ and $x=\frac{1}{3}$ if $\int\limits_{-1}^1 f dx = \frac{14}{3}$, find the cubic $f(x)$.\\
\hfill{\textbf{(1992 - 4 Marks)}}

\item What normal to the curve $y=x^2$ form the shortest chord?\\
\hfill (1992 - 6 Marks)

\item Find the equation of the normal to the curve $y=(1+x)^y + \sin^{-1}(\sin^2x)$ at $x=0$

\hfill (1993 - 3 Marks)


\item Let 
\[ f(x) = \begin{cases}
-x^3 + \brak{\frac{b^3-b^2+b-1}{b^2+3b+2}}, & \text{if } 0\leq x < 1\\
2x-3, & \text{if } 1\leq x \leq 3
\end{cases}
\]

\hfill (1993 - 5 Marks)

Find all possible real values of $b$ such that $f(x)$ has the smallest value at $x=1$.

\item The curve $y = ax^3 + bx^2 + cx + 5$, touches the $x-axis$ at $P(-2,0)$ and cuts the $y-axis$ at a point $Q$, where its gradient is 3. Find $a, b, c$.

\hfill (1994 - 5 Marks)

\item The circle $x^2 + y^2 = 1$ cuts the $x-axis$ at $P$ and $Q$. Another circle with centre at $Q$ and variable radius intersects the first circle at $R$ above the $x-axis$ and the line segment $PQ$ $S$. Find the maximum area of the triangle $QSR$.

\hfill (1994 - 5 Marks)


\item Let $\brak{h,k}$ be a fixed point, where $h>0, k>0$. A straight line passing through this point cuts the positive direction of the coordinate axes at the points $P$ and $Q$. Find the minimum area of the triangle $OPQ$, $O$ being the origin.

\hfill{(1995 - 5 Marks)}


\item A curve $y=f(x)$ passes through the point $P(1,1)$. The normal to the curve at $P$ is a $(y-1) + (x-1) = 0$. If the slope of the tangent at any point on the curve is proportional to the the ordinate of the point, determine the equation of the curve. Also obtain the area bounded by the $y-axis$, the curve and the normal to the curve at $P$.\\
\hfill (1996 - 5 Marks)

\item Determine the points of maxima and minima of the function $f(x) = \frac{1}{8}\ln{x} - bx + x^2, x>0$, where $b \geq 0$ is a constant.

\hfill (1996 - 5 Marks)


\item Let 
\[ f(x) = \begin{cases}
xe^{ax}, & \text{if } x \leq 0\\
x+ax^2 - x^3, & \text{if } x > 0
\end{cases}\\
\]

Where $a$ is a positive constant. Find the interval in which $f'(x)$ is increasing.\\
\hfill (1996 - 3 Marks)

\item Let $a + b = 4$, where $a<2$, and let $g(x)$ be a differentiable function.\\\\
If $\frac{dg}{dx}>0$ for all x, prove that $\int\limits_0^a g(x) dx + \int\limits_0^b g(x) dx$ increase as $(b-a)$ increases.

\hfill (1997 - 5 Marks)


\item Suppose $f(x)$ is a function satisfying the following conditions\\
\begin{enumerate}[label = (\alph*)]
\item $f(0) = 2,\ f(1) = 1$
\item $f$ has a minimum value at $x = \frac{5}{2}$, and
\item for all $x$,
\end{enumerate}
$$f'(x)=
\begin{vmatrix}
2ax & 2ax-1 & 2ax+b+1\\
b & b+1 & -1\\
2(ax+b) & 2ax+2b+1 & 2ax+b
\end{vmatrix}$$

\hfill (1998 - 8 Marks)

where $a, b$ are some constants. Determine the constants $a, b$ and the function $f(x)$.

\item A curve $C$ has the property that if the tangent drawn at any point $P$ on $C$ meets the co-ordinate axes at $A$ and $B$, then $P$ is the mid-point of $AB$. The curve passes through the point \brak{1,1}. Determine the equation of the curve.

\hfill (1998 - 8 Marks)

 
\end{enumerate}
\end{document}