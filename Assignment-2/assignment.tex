%iffalse
\let\negmedspace\undefined
\let\negthickspace\undefined
\documentclass[journal,12pt,twocolumn]{IEEEtran}
\usepackage{cite}
\usepackage{amsmath,amssymb,amsfonts,amsthm}
\usepackage{algorithmic}
\usepackage{graphicx}
\usepackage{textcomp}
\usepackage{xcolor}
\usepackage{txfonts}
\usepackage{listings}
\usepackage{enumitem}
\usepackage{mathtools}
\usepackage{gensymb}
\usepackage{comment}
\usepackage[breaklinks=true]{hyperref}
\usepackage{tkz-euclide} 
\usepackage{listings}
\usepackage{gvv}                                        
%\def\inputGnumericTable{}                                 
\usepackage[latin1]{inputenc}                                
\usepackage{color}                                            
\usepackage{array}                                            
\usepackage{longtable}                                       
\usepackage{calc}                                             
\usepackage{multirow}                                         
\usepackage{hhline}                                           
\usepackage{ifthen}                                           
\usepackage{lscape}
\usepackage{tabularx}
\usepackage{array}
\usepackage{float}


\newtheorem{theorem}{Theorem}[section]
\newtheorem{problem}{Problem}
\newtheorem{proposition}{Proposition}[section]
\newtheorem{lemma}{Lemma}[section]
\newtheorem{corollary}[theorem]{Corollary}
\newtheorem{example}{Example}[section]
\newtheorem{definition}[problem]{Definition}
\newcommand{\BEQA}{\begin{eqnarray}}
\newcommand{\EEQA}{\end{eqnarray}}
\newcommand{\define}{\stackrel{\triangle}{=}}
\theoremstyle{remark}
\newtheorem{rem}{Remark}

% Marks the beginning of the document
\begin{document}
\bibliographystyle{IEEEtran}
\vspace{3cm}

\title{CHAPTER - 13\\Properties of Triangles}
\author{EE24BTECH11061 - Rohith Sai}
\maketitle
\newpage
\bigskip

\renewcommand{\thefigure}{\theenumi}
\renewcommand{\thetable}{\theenumi}

\section{C. MCQs with One Correct Answer}

\begin{enumerate}
\item In a triangle $ABC$, $\angle B = \frac{\pi}{3}$ and $\angle C = \frac{\pi}{4}$. Let $D$ divide $BC$ internally in the ratio $1:3$ then $\frac{\sin\angle BAD}{\sin \angle CAD}$ is equal to
\begin{enumerate}
\item $\frac{1}{\sqrt6}$
\item $\frac{1}{3}$
\item $\frac{1}{\sqrt3}$
\item $\sqrt{\frac{2}{3}}$
\end{enumerate}
\hfill (1995S)

\item In a triangle $ABC$, $2ac\sin\frac{1}{2}(A-B+C) = $
\begin{enumerate}
\item $a^2 + b^2 - c^2$
\item $c^2 + a^2 - b^2$
\item $b^2 - c^2 - a^2$
\item $c^2 - a^2 - b^2$
\end{enumerate}
\hfill (2000S)

\item In a triangle $ABC$, let $\angle C = \frac{\pi}{2}$. If $r$ is the inradius and $R$ is the circumradius of the triangle, then $2(r+R)$ is equal to
\begin{enumerate}
\item $a+b$
\item $b+c$
\item $c+a$
\item $a+b+c$
\end{enumerate}
\hfill (2000S)

\item A pole stands vertically inside a triangular park $\triangle ABC$. If the angle of elevation of the top of the pole from each corner of the park is same, then in $\triangle ABC$ the foot of the pole is at the
\begin{enumerate}
\item centroid
\item circumcentre
\item incentre
\item orthocentre
\end{enumerate}
\hfill (2000S)

\item A man from the top of a $100$ metres high tower sees a car moving towards the tower at an angle of depression of $30\degree$. After some time, the angle of depression becomes $60\degree$. The distance (in metres) travelled by the car during this time is
\begin{enumerate}
\item $100\sqrt{3}$
\item $\frac{200\sqrt{3}}{3}$
\item $\frac{100\sqrt{3}}{3}$
\item $200\sqrt{3}$
\end{enumerate}
\hfill (2001S)

\item Which of the following pieces of data does NOT uniquely determine an acute-angled triangle $\triangle ABC$ ($R$ being the radius of the circumcircle)?
\begin{enumerate}
\item $a, \sin A, \sin B$
\item $a, b, c$
\item $a, \sin B, R$
\item $a, \sin A, R$
\end{enumerate}
\hfill (2002S)

\item If the angles of a triangle are in the ratio $4:1:1$, then the ratio of the longest side to the perimeter is
\begin{enumerate}
\item $\sqrt{3}:2+\sqrt{3}$
\item $1:6$
\item $1:2+\sqrt{3}$
\item $2:3$
\end{enumerate}
\hfill (2003S)

\item The sides of a triangle are in the ratio $1:\sqrt{3}:2$, then the angles of the triangle are in the ratio
\begin{enumerate}
\item $1:3:5$
\item $2:3:4$
\item $3:2:1$
\item $1:2:3$
\end{enumerate}
\hfill (2004S)

\item In an equilateral triangle, $3$ coins of radii $1$ unit each are kept so they touch each other and also the sides of the triangle. Area of the triangle is 
\begin{figure}
    \centering
    \includegraphics[width=0.5\linewidth]{figs/WhatsApp Image 2024-08-07 at 23.47.19.jpeg}
    \label{fig:enter-label}
\end{figure}
\begin{enumerate}
\item $4+2\sqrt{3}$
\item $6+4\sqrt{3}$
\item $12+\frac{7\sqrt{3}}{4}$
\item $3+\frac{7\sqrt{3}}{4}$
\end{enumerate}
\hfill (2005S)

\item In a triangle $ABC, a, b, c$  are the lengths of its sides and $A, B, C$ are the angles of triangle $ABC$. The correct relation is given by
\begin{enumerate}
\item $(b-c)\sin \brak{\frac{B-C}{2}} = a \cos \brak{\frac{A}{2}}$
\item $(b-c) \cos \frac{A}{2} = a \sin \brak{\frac{B-C}{2}}$
\item $(b+c)\sin \brak{\frac{B+C}{2}} = a \cos \brak{\frac{A}{2}}$
\item $(b-c) \cos \frac{A}{2} = a \sin \brak{\frac{B+C}{2}}$
\end{enumerate}
\hfill (2005S)

\item One angle of an isosceles $\triangle$ is $120\degree$ and radius of its incircle $= \sqrt{3}$. Then the area of the triangle in sq. units is 
\begin{enumerate}
\item $7+12\sqrt{3}$
\item $12-7\sqrt{3}$
\item $12+7\sqrt{3}$
\item $4\pi$
\end{enumerate}
\hfill (2006 - 3M, -1)

\item Let $ABCD$ be a quadrilateral with area $18$, with side $AB$ parallel to the side $CD$ and $2AB = CD$. Let $AD$ be perpendicular to $AB$ and $CD$. If a circle is drawn inside the quadrilateral $ABCD$ touching all the sides, then the radius is
\begin{enumerate}
\item $3$
\item $2$
\item $\frac{3}{2}$
\item $1$
\end{enumerate}
\hfill (2007 - 3 Marks)

\item If the angles $A, B$ and $C$ of a triangle are in an arithmetic progression and if $a, b and c$ denote the lengths of the sides opposite to $A, B$ and $C$ respectively, then the value of the expression $\frac{a}{c}\sin 2C + \frac{c}{a} \sin 2A$ is
\begin{enumerate}
\item $\frac{1}{2}$
\item $\frac{\sqrt{3}}{2}$
\item $1$
\item $\sqrt{3}$
\end{enumerate}
\hfill (2010)

\item Let $PQR$ be a triangle of area $\Delta$ with $a=2, b= \frac{7}{2}$ and $c=\frac{5}{2}$, where $a, b$ and $c$ are the lengths of the sides of the triangle opposite to the angles at $P, Q$ and $R$ respectively. Then $\frac{2\sin P - \sin 2P}{2\sin P + \sin 2P}$ equals
\begin{enumerate}
\item $\frac{3}{4\Delta}$
\item $\frac{45}{4\Delta}$
\item $\brak{\frac{3}{4\Delta}}^2$
\item $\brak{\frac{45}{4\Delta}}^2$
\end{enumerate}
\hfill (2012)

\item In a triangle the sum of two sides is $x$ and the product of the same sides is $y$. If $x^2-c^2=y$, where $c$ is the third side od the triangle, then the ratio of the inradius to the circum-radius of the triangle is
\begin{enumerate}
\item $\frac{3y}{2x(x+c)}$
\item $\frac{3y}{2c(x+c)}$
\item $\frac{3y}{4x(x+c)}$
\item $\frac{3y}{4c(x+c)}$
\end{enumerate}
\hfill (JEE Adv. 2014)

\end{enumerate}

\end{document}
