\let\negmedspace\undefined
\let\negthickspace\undefined
\documentclass[journal]{IEEEtran}
\usepackage[a5paper, margin=10mm, onecolumn]{geometry}
%\usepackage{lmodern} % Ensure lmodern is loaded for pdflatex
\usepackage{tfrupee} % Include tfrupee package

\setlength{\headheight}{1cm} % Set the height of the header box
\setlength{\headsep}{0mm}     % Set the distance between the header box and the top of the text

\usepackage{gvv-book}
\usepackage{gvv}
\usepackage{cite}
\usepackage{amsmath,amssymb,amsfonts,amsthm}
\usepackage{algorithmic}
\usepackage{graphicx}
\usepackage{textcomp}
\usepackage{xcolor}
\usepackage{txfonts}
\usepackage{listings}
\usepackage{enumitem}
\usepackage{mathtools}
\usepackage{gensymb}
\usepackage{comment}
\usepackage[breaklinks=true]{hyperref}
\usepackage{tkz-euclide} 
\usepackage{listings}
% \usepackage{gvv}                                        
\def\inputGnumericTable{}                                 
\usepackage[latin1]{inputenc}                                
\usepackage{color}                                            
\usepackage{array}                                            
\usepackage{longtable}                                       
\usepackage{calc}                                             
\usepackage{multirow}                                         
\usepackage{hhline}                                           
\usepackage{ifthen}                                           
\usepackage{lscape}
\begin{document}
\bibliographystyle{IEEEtran}
\vspace{3cm}

\title{CHAPTER - 1\\Vector Arithmetic}
\author{EE24BTECH11061 - Rohith Sai}
\maketitle

\renewcommand{\thefigure}{\theenumi}
\renewcommand{\thetable}{\theenumi}

\section{1.2 Point Vectors}
\begin{enumerate}
\item [1.2.27] In a harbour, wind is blowing at the speed of $72$ km/h and the flag on the mast of a boat anchored in the harbour flutters along the N-E direction. If the boat starts moving at a speed of $51$ km/h to the north, what is the direction of the flag on the
mast of the boat ?\\
\textbf{Solution:}
The wind velocity $\vec{w}$ is blowing at 72 km/h towards the north-east direction. In terms of components, we can write:
\begin{align}
    \vec{w} = 72 \myvec{
    \frac{1}{\sqrt{2}} \\
    \frac{1}{\sqrt{2}}}
\end{align}
The boat velocity $\vec{b}$ is moving at 51 km/h towards the north. In vector form, this is:
\begin{align}
    \vec{b} = 51 \myvec{
    0 \\
    1}
\end{align}
The wind vector $\vec{w}$ and boat vector $\vec{b}$ are given by:

\begin{align}
\vec{w} = \myvec{
\frac{72}{\sqrt{2}} \\
\frac{72}{\sqrt{2}}}=\myvec{36\sqrt{2}\\36\sqrt{2}}
\end{align}

\begin{align}
\mathbf{b} = \myvec{
0 \\
51}
\end{align}

The resultant vector $\vec{v}$ is calculated as:

\begin{align}
\vec{v} = \vec{w} - \vec{b} = \myvec{
36\sqrt{2} \\
36\sqrt{2} - 51
}
\end{align}

The x and y components of $\vec{c}$ are $\vec{v_x}$ and $\vec{v_y}$ respectively:
\begin{align}
    \vec{v_x} = 36\sqrt{2}
\end{align}
\begin{align}
    \vec{v_y} = 36\sqrt{2}-51
\end{align}

Let $\theta$ be the angle made by $\vec{v}$ with $\vec{v_x}$:
\begin{align}
    \tan \theta = \frac{\vec{v_y}}{\vec{v_x}}
\end{align}
\begin{align}
    \tan \theta = \frac{36\sqrt{2}-51}{36\sqrt{2}}
\end{align}

Thus, the direction of the flag on the mast of the boat is:
\begin{align}
    \theta = \arctan\brak{\frac{36\sqrt{2}-51}{36\sqrt{2}}}
\end{align}
\end{enumerate}
\end{document}